\documentclass{../../shared/survey}

\title{Страдаю ли я посттравматическим стрессовым расстройством? (ПТСР)}

\fancyhead[L]{\textbf{Посттравматическое стрессовое расстройство}}
\fancyfoot[L]{Предоставлено \textbf{MentalDesk}}
\fancyfoot[R]{\footnotesize{© 2023 MentalDesk.ru}}

\begin{document}

\section*{\huge{Страдаю ли я посттравматическим стрессовым \\ расстройством? (ПТСР)}}

\begin{flushleft}
Данный опросник создан с целью помочь клиентам оценить свой опыт и понять,
требует ли он дальнейшей проработки со специалистом.
\end{flushleft}

% Survey 
\yesno{Испытывали ли Вы что-то пугающее, ужасное или травмирующее?  Например, попадали ли Вы в аварию? Подвергались ли физическому или сексуализированному насилию?}

\QItem{ \Qq{Если Вы ответили да, то ответьте, пожалуйста, на следующие вопросы о своем состоянии в течении прошлого месяца.} \Qlines{0} }

\yesnoa[a]{Видели ли Вы кошмары об этом событии? Или думали о нем, когда не хотели?}

\yesnoa[b]{Старались ли Вы не думать об этом событии? Избегали ли ситуаций, которые напоминали о случившемся?}

\yesnoa[c]{Были ли Вы настороже или легко пугались?}

\yesnoa[d]{Чувствовали ли Вы отстранённость от людей, любимой деятельности?}

\yesnoa[e]{Ощущали ли Вы себя виновным или не могли перестать винить себя или других за происшедшее с Вами событие или его последствия?}

\begin{flushleft}
Если Вы ответили «да» на первый вопрос и на три или более вопросов
далее, то Вы страдаете от посттравматического стрессового расстройства.
Пожалуйста, обратитесь к специалисту, чтобы начать путь к излечению. Или
если он у Вас есть, начните работать над этим с ним.
\end{flushleft}

\end{document}

