\documentclass[a4paper,12pt]{article}
\usepackage[T2A]{fontenc}
\usepackage[utf8]{inputenc}
\usepackage[english,russian]{babel}
\usepackage{paratype}
\usepackage[
includehead,
% showframe,
lmargin=32px,
rmargin=32px,
tmargin=0px,
bmargin=48px
]{geometry}
\usepackage{fancyhdr}
\usepackage{wasysym}% provides \ocircle and \Box
\usepackage{enumitem}% easy control of topsep and leftmargin for lists
\usepackage{color}% used for background color
\usepackage{forloop}% used for \Qrating and \Qlines
\usepackage{ifthen}% used for \Qitem and \QItem


%%%%%%%%%%%%%%%%%%%%%%%%%%%%%%%%%%%%%%%%%%%%%%%%%%%%%%%%%%%%
%% Beginning of questionnaire command definitions %%
%%%%%%%%%%%%%%%%%%%%%%%%%%%%%%%%%%%%%%%%%%%%%%%%%%%%%%%%%%%%
%%
%% 2010, 2012 by Sven Hartenstein
%% mail@svenhartenstein.de
%% http://www.svenhartenstein.de
%%
%% Please be warned that this is NOT a full-featured framework for
%% creating (all sorts of) questionnaires. Rather, it is a small
%% collection of LaTeX commands that I found useful when creating a
%% questionnaire. Feel free to copy and adjust any parts you like.
%% Most probably, you will want to change the commands, so that they
%% fit your taste.
%%
%% Also note that I am not a LaTeX expert! Things can very likely be
%% done much more elegant than I was able to. If you have suggestions
%% about what can be improved please send me an email. I intend to
%% add good tipps to my website and to name contributers of course.
%%
%% 10/2012: Thanks to karathan for the suggestion to put \noindent
%% before \rule!

%% \Qq = Questionaire question. Oh, this is just too simple. It helps
%% making it easy to globally change the appearance of questions.
\newcommand{\Qq}[1]{#1}

%% \QO = Circle or box to be ticked. Used both by direct call and by
%% \Qrating and \Qlist.
\newcommand{\QO}{$\Circle$}% or: $\ocircle$

%% \Qrating = Automatically create a rating scale with NUM steps, like
%% this: 0--0--0--0--0.
\newcounter{qr}
\newcommand{\Qrating}[1]{\QO\forloop{qr}{1}{\value{qr} < #1}{---\QO}}

%% \Qline = Again, this is very simple. It helps setting the line
%% thickness globally. Used both by direct call and by \Qlines.
\newcommand{\Qline}[1]{\noindent\rule{#1}{0.6pt}}

%% \Qlines = Insert NUM lines with width=\linewith. You can change the
%% \vskip value to adjust the spacing.
\newcounter{ql}
\newcommand{\Qlines}[1]{\forloop{ql}{0}{\value{ql}<#1}{\vskip0em\Qline{\linewidth}}}

%% \Qlist = This is an environment very similar to itemize but with
%% \QO in front of each list item. Useful for classical multiple
%% choice. Change leftmargin and topsep accourding to your taste.
\newenvironment{Qlist}{%
\renewcommand{\labelitemi}{\QO}
\begin{itemize}[leftmargin=1.5em,topsep=-.5em]
}{%
\end{itemize}
}

%% \Qtab = A "tabulator simulation". The first argument is the
%% distance from the left margin. The second argument is content which
%% is indented within the current row.
\newlength{\qt}
\newcommand{\Qtab}[2]{
\setlength{\qt}{\linewidth}
\addtolength{\qt}{-#1}
\hfill\parbox[t]{\qt}{\raggedright #2}
}

%% \Qitem = Item with automatic numbering. The first optional argument
%% can be used to create sub-items like 2a, 2b, 2c, ... The item
%% number is increased if the first argument is omitted or equals 'a'.
%% You will have to adjust this if you prefer a different numbering
%% scheme. Adjust topsep and leftmargin as needed.
\newcounter{itemnummer}
\newcommand{\Qitem}[2][]{% #1 optional, #2 notwendig
\ifthenelse{\equal{#1}{}}{\stepcounter{itemnummer}}{}
\ifthenelse{\equal{#1}{a}}{\stepcounter{itemnummer}}{}
\begin{enumerate}[topsep=2pt,leftmargin=2em]
\item[\textbf{\arabic{itemnummer}#1.}] #2
\end{enumerate}
}

%% \QItem = Like \Qitem but with alternating background color. This
%% might be error prone as I hard-coded some lengths (-5.25pt and
%% -3pt)! I do not yet understand why I need them.
\definecolor{bgodd}{rgb}{0.8,0.8,0.8}
\definecolor{bgeven}{rgb}{0.9,0.9,0.9}
\newcounter{itemoddeven}
\newlength{\gb}
\newcommand{\QItem}[2][]{% #1 optional, #2 notwendig
\setlength{\gb}{\linewidth}
\setlength{\fboxsep}{8pt}
\addtolength{\gb}{-5.25pt}
\ifthenelse{\equal{\value{itemoddeven}}{0}}{%
\noindent\colorbox{bgeven}{\hskip-3pt\begin{minipage}{\gb}\Qitem[#1]{#2}\end{minipage}}%
\stepcounter{itemoddeven}%
}{%
\noindent\colorbox{bgodd}{\hskip-3pt\begin{minipage}{\gb}\Qitem[#1]{#2}\end{minipage}}%
\setcounter{itemoddeven}{0}%
}
}
% \adjustbox{cframe=<color>}{<text>}

%%%%%%%%%%%%%%%%%%%%%%%%%%%%%%%%%%%%%%%%%%%%%%%%%%%%%%%%%%%%
%% End of questionnaire command definitions %%
%%%%%%%%%%%%%%%%%%%%%%%%%%%%%%%%%%%%%%%%%%%%%%%%%%%%%%%%%%%%



\begin{document}
% \setlength{\hsize}{0.9\hsize}% emphasize effects
\title{Страдаю ли я посттравматическим стрессовым расстройством? (ПТСР)}
\author{Mental Desk}
% \date{}
\pagestyle{fancy}
\usefont{T2A}{PTSans-TLF}{m}{n}
% \maketitle

% Header 
\fancyhf{} % clear existing header/footer entries
\renewcommand{\headrulewidth}{0pt}
% \fancyhead[L]{\textbf{Страдаю ли я посттравматическим стрессовым расстройством? (ПТСР)}}

% Body
\section*{\huge{Страдаю ли я посттравматическим стрессовым расстройством? (ПТСР)}}
\begin{flushleft}
Данный опросник создан с целью помочь клиентам оценить свой опыт и понять,
требует ли он дальнейшей проработки со специалистом.
\end{flushleft}

% Survey 
\QItem{ \Qq{Испытывали ли Вы что-то пугающее, ужасное или травмирующее?  Например, попадали ли Вы в аварию? Подвергались ли физическому или сексуализированному насилию?}
\begin{Qlist}
\item Да
\item Нет
\end{Qlist}
}
\vskip.4em
\QItem{ \Qq{Если Вы ответили да, то ответьте, пожалуйста, на следующие вопросы о своем состоянии в течении прошлого месяца.} \Qlines{0} }
\vskip.4em
\QItem[a]{ \Qq{Видели ли Вы кошмары об этом событии? Или думали о нем, когда не хотели?}
\begin{Qlist}
\item Да
\item Нет
\end{Qlist}
}
\vskip.4em
\QItem[b]{ \Qq{Старались ли Вы не думать об этом событии? Избегали ли ситуаций, которые напоминали о случившемся?}
\begin{Qlist}
\item Да
\item Нет
\end{Qlist}
}
\vskip.4em
\QItem[c]{ \Qq{Были ли Вы настороже или легко пугались?}
\begin{Qlist}
\item Да
\item Нет
\end{Qlist}
}
\vskip.4em
\QItem[d]{ \Qq{Чувствовали ли Вы отстранённость от людей, любимой деятельности?}
\begin{Qlist}
\item Да
\item Нет
\end{Qlist}
}
\vskip.4em
\QItem[e]{ \Qq{Ощущали ли Вы себя виновным или не могли перестать винить себя или других за происшедшее с Вами событие или его последствия?}
\begin{Qlist}
\item Да
\item Нет
\end{Qlist}
}

\begin{flushleft}
Если Вы ответили «да» на первый вопрос и на три или более вопросов
далее, то Вы страдаете от посттравматического стрессового расстройства.
Пожалуйста, обратитесь к специалисту, чтобы начать путь к излечению. Или
если он у Вас есть, начните работать над этим с ним.
\end{flushleft}

% Footer
\fancyfoot[L]{Предоставлено \textbf{MentalDesk}}
\fancyfoot[R]{\footnotesize{© 2023 MentalDesk.ru}}

\end{document}
