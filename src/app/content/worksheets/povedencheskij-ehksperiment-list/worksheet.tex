\documentclass{../../shared/survey}

\usepackage{tcolorbox}
\usepackage{tikz}
\fancyhead[L]{\textbf{Поведенческий эксперимент (вертикальный формат)}}

\makeatletter
\newcommand{\DrawLine}{%
  \begin{tikzpicture}
  \path[use as bounding box] (0,0) -- (\linewidth,0);
  \draw[color=black!75!black,dashed,dash phase=2pt]
        (0-\kvtcb@leftlower-\kvtcb@boxsep,0)--
        (\linewidth+\kvtcb@rightlower+\kvtcb@boxsep,0);
  \end{tikzpicture}%
  }
\makeatother

\begin{document}

\section*{\huge{Поведенческий эксперимент}}

\begin{flushleft}
\large{\textbf{Часть 1:} Прогноз}
\end{flushleft}


\begin{tcolorbox}[standard jigsaw, opacityback=0, top=0]


\vspace{.5\baselineskip}
\parbox[][4cm][t]{18cm}{
{\textbf{\large Мысли}} \\
Какую мысль или убеждение вы бы хотели проверить?
}
\DrawLine

\vspace{.5\baselineskip}
\parbox[][4cm][t]{18cm}{
{\textbf{\large Эксперимент}} \\
Как вы узнаете, что ваш эксперимент удался?
}

\DrawLine

\vspace{.5\baselineskip}
\parbox[][4cm][t]{18cm}{
{\textbf{\large Прогноз}} \\
Каков ваш прогноз? \\
Как вы узнаете, что ваш эксперимент удался?
\begin{tcolorbox}[standard jigsaw, enlarge by=-4mm, height=2.8cm, width=6cm, boxsep=4mm, flush right]
\center{Какие ощущения вы ожидаете после эксперимента?}
\end{tcolorbox}
}

\end{tcolorbox}


\begin{flushleft}
\large{\textbf{Часть 2:} Результат}
\end{flushleft}


\begin{tcolorbox}[standard jigsaw,opacityback=0, top=0]

\vspace{.5\baselineskip}
\parbox[][4cm][t]{18cm}{
{\textbf{\large Результат}} \\
Что произошло во время эксперимента? \\
\begin{tcolorbox}[standard jigsaw, enlarge by=-7mm, height=2.8cm, width=6cm, boxsep=4mm, flush right]
\center{Какие ощущения вы испытали после эксперимента?}
\end{tcolorbox}
}

\DrawLine

\vspace{.5\baselineskip}
\parbox[][4cm][t]{18cm}{
{\textbf{\large Новые идеи}} \\
Чему вы научились? \\
Учитывая данные эксперимента, каковы ваши новые идеи?}

\end{tcolorbox}

\end{document}
