%%% Local Variables:
%%% mode: latex
%%% TeX-master: t
%%% End:
\documentclass[a4paper,12pt,oneside,headsepline]{scrartcl}
\usepackage[T2A]{fontenc}
\usepackage[utf8]{inputenc}
\usepackage[english,russian]{babel}
\usepackage{paratype}
\usepackage{parskip}
\usepackage[most]{tcolorbox}
\usepackage{color}
\usepackage[margin=1.2cm]{geometry}
\usepackage{lmodern}
\usepackage{tikz}
\tcbuselibrary{skins,breakable}

% Colors
\definecolor{primary}{RGB}{100, 26, 230}
\definecolor{secondary}{RGB}{217, 38, 169}
\definecolor{accent}{RGB}{31, 178, 166}
\definecolor{neutral}{RGB}{42, 50, 60}
\definecolor{base}{RGB}{29, 35, 42}
\definecolor{info}{RGB}{58, 191, 248}
\definecolor{success}{RGB}{54, 211, 153}
\definecolor{warning}{RGB}{251, 189, 35}
\definecolor{error}{RGB}{248, 114, 114}

\tcbset{frogbox/.style={enhanced,colback=green!10,colframe=green!65!black,
enlarge top by=5.5mm,
overlay={\foreach \x in {2cm,3.5cm} {
\begin{scope}[shift={([xshift=\x]frame.north west)}]
\path[draw=green!65!black,fill=green!10,line width=1mm] (0,0) arc (0:180:5mm);
\path[fill=black] (-0.2,0) arc (0:180:1mm);
\end{scope}}}}}

\newtcolorbox{mybox}[1][]{enlarge left by=-1.2cm,enhanced, width=21cm,
colback=neutral!5!white,
colbacktitle=neutral!85!black!50!white,
boxrule=0pt,frame hidden,fonttitle=\bfseries,watermark color=yellow!50!white,
boxsep=1mm,left=1.2cm,right=1.2cm,top=4mm,bottom=4mm,
underlay={\begin{tcbclipinterior}
% \draw[white!40!white,line width=.5cm] (interior.south west) -- (interior.north east);
\end{tcbclipinterior}},
attach boxed title to top center={yshift=-2mm},#1}


\title{Дневник Активности}
\author{Mental Desk}

\begin{document}

\titlehead{
% \begin{mybox}[]
% \begin{tcbraster}[raster columns=2, raster equal height=rows, 
% size=small,boxrule=0pt,frame hidden, bicolor, colbacklower=white]
% \begin{tcolorbox}
%   {\Large Название Вашего Центра \\ }
%   {Психотерапевт: Фамилия Имя Отчество \\}
%   {\emph{\Large Дневник Активности}} \\
% \end{tcolorbox}
% \begin{tcolorbox}
%   {\small Клиент: Фамилия Имя Отчество \\}
%   {\small Дата: 29/10/2023 \\} 
% \end{tcolorbox}
% \end{tcbraster}
% \end{mybox}
\begin{mybox}[]

\begin{tcolorbox}[size=small,boxrule=0pt,frame hidden, bicolor, colbacklower=white]
  {\Large Название Вашего Центра \hfill  \small Клиент: Фамилия Имя Отчество \\}
  \vspace*{2mm}
  {\small Психотерапевт: Фамилия Имя Отчество \hfill \small Дата: 29/10/2023 \\} 
  {\emph{\Large Дневник Активности} \\} 
\end{tcolorbox}
\end{mybox}
}
\subject{Dissertation}
\title{Digital space simulation with the DSP\,56004}
\subtitle{Short but sweet?}
\author{Fuzzy George}
\date{30. February 2002}
\publishers{Adviser Prof. John Eccentric Doe}
\maketitle




\begin{tcolorbox}[enhanced,remember,colback=red!5!white,colframe=red!75!black,
fonttitle=\bfseries,title=The four corners of a paper,
overlay={\begin{tcbclipinterior}
\draw[red!40!white,line width=1cm] (interior.south west)--(interior.north east);
\end{tcbclipinterior}}
]
This is a tcolorbox.
\end{tcolorbox}

\begin{tcolorbox}[]
{\Large Название Вашего Центра \\}
\end{tcolorbox}


\begin{tcolorbox}[enhanced,title=My Title,
overlay={\begin{tcbclipinterior}
\draw[red,line width=1cm] (interior.north west)--(interior.south east);
\draw[red,line width=1cm] (interior.south west)--(interior.north east);
\end{tcbclipinterior}}]

adsf
\end{tcolorbox}



\raggedright
Самоконтроль – важный навык, которому стоит научиться каждому клиенту, проходящему когнитивно-поведенческую терапию. Этот дневник – удобный инструмент для самомониторинга при обсессивно-компульсивном расстройстве (ОКР). Он включает в себя пять столбцов для фиксирования триггеров, навязчивых идей, эмоций и компульсий.

\begin{Large}Описание\end{Large}

\raggedright Самоконтроль – важный навык, которому стоит научиться каждому клиенту,
проходящему когнитивно-поведенческую терапию. Этот дневник – удобный
инструмент для самомониторинга при обсессивно-компульсивном расстройстве
(ОКР). Он включает в себя пять столбцов для фиксирования триггеров,
навязчивых идей, эмоций и компульсий.

\mbox{\begin{Large}Инструкция.\end{Large}}

\raggedright Тщательно проинструктируйте клиента, как ему необходимо вести
дневник.  Для удобства в нем есть самонаводящие вопросы. Можно выделить время на
сессии, чтобы заполнить его вместе, пока клиент не почувствует уверенность в
работе с дневником.

\end{document}
