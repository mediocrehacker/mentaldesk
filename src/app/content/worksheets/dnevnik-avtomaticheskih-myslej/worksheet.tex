\documentclass{worksheet_landscape}

\title{Дневник Автоматических мыслей}
\CustomHeader{Дневник Автоматических мыслей}


\begin{document}

% Table 
\begin{table}[]
\begin{center}
\renewcommand{\arraystretch}{1.8}
\newcolumntype{Y}{>{\raggedright\arraybackslash}X}
\begin{tabularx}{\textwidth}{|Y|Y|Y|Y|Y|Y|Y|}

\hline
\textbf{Дата и время} & \textbf{Ситуация} & {\textbf{Автоматические мысли}} & \textbf{Эмоции} & \textbf{Когнитивное искажение} & \textbf{Адаптивный ответ} & \textbf{Результат} \\
\hline
&
\vspace{12.7cm}
\small{\textit{Опишите событие, которое вызывает неприятные эмоции или физические
ощущения.}} &
\vspace{12.7cm}
\small{\textit{Запишите мысли, которые Вас посещают. Оцените веру в эти мысли по
    шкале от 0 до 100.}} &
\vspace{11.3cm}
\small{\textit{Укажите, какие эмоции вызывают эти мысли. Например, злость,
    тревога, грусть. Оцените по 100\% шкале их интенсивность.}} &
\vspace{8cm}
\small{\textit{Катострофизация, мышление "все или ничего", ошибка планирования,
    персонализация, утверждения "я должен", иллюзия прозрачности, эффект первого
    впечатления, негативный фильтр, склонность к подтверждению, эмоциональное
    обоснование и т.д.}} &
\vspace{12.3cm}
\small{\textit{Нейтральный ответ на автоматические мысли. То, как бы Вы ответили
    и поддержали близкого человека.}} &
\vspace{11.8cm}
\small{\textit{Опищите свои эмоции теперь. Отметьте, изменилась ли интенсивность
    эмоций по шкале от 0 до 100\%}} \\[435pt]
\hline
\end{tabularx}
\end{center}
\label{tab:activity_diary}
\end{table}

\end{document}
